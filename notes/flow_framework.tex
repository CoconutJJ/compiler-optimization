\documentclass{article}
\usepackage[]{amsmath}
\usepackage{amssymb}
\usepackage[margin=0.5in]{geometry}
\author{David Yue}
\title{Compiler Optimization Notes}
\date{}
\setlength{\parindent}{0pt}
\begin{document}
\maketitle
\section*{Partial Order of a Semi lattice}
Definition of \(x \leq y\)
\begin{equation*}
    x \leq y \iff x \wedge y = x
\end{equation*}
The \(\leq\) operator satisfies 3 properties
\begin{enumerate}
    \item Reflexivity: \(x \leq x \) for all \(x\)
    \item Anti-Symmetry: \(x \leq y\) and \(y \leq x\) implies \(x = y\)
    \item Transitivity: \(x \leq y\) and \(y \leq z\) implies \(x \leq z\)
\end{enumerate}

\section*{Greatest Lower Bound}

\(z\) is the greatest lower bound of \(x\) if \(z \leq x\) and for every \(y\),
\(y \leq z\) implies \(y = z\)

\begin{equation}
    x \wedge y \text{ is the greatest lower bound of } x \text{ and } y
\end{equation}

Proof:

Let \(g = x \wedge y\). Suppose by contradiction that there exists some
element \(z\) such that \(z \leq x\), \(z \leq y\) and \(g \leq z\).

Consider
\begin{align*}
    z \wedge g & = z \wedge (x \wedge y)                                       \\
               & = (z \wedge x) \wedge y \quad (\wedge \text{ is associative}) \\
               & = z \wedge y \quad (z \leq x \text{ is by hypothesis })       \\
               & = z  \quad (z \leq y \text{ is by hypothesis})
\end{align*}

By definition of the partial order operator,

\begin{equation}
    z \wedge g = z \implies z \leq g
\end{equation}

This is a contradiction to our hypothesis that \(g \leq z\).

\section*{Transfer Functions}

Let \(F: V \rightarrow V\) be the set of transfer functions in a data flow
framework. The set \(F\) has the following properties:

\begin{enumerate}
    \item \(F\) has an identity function \(I\) such that \(I(x) = x\) for all \(x \in V\)
    \item \(F\) is closed under composition. Let \(f,g \in F\), then \(f(g) \in F\)
\end{enumerate}

\section*{Monotone Frameworks}

A data flow framework is said to be monotone, if for every \(x,y \in V\) and \(f \in F\)

\begin{equation}
    x \leq y \text{ implies } f(x) \leq f(y)
\end{equation}

or equivalently

\begin{equation}
    f(x \wedge y) \leq f(x) \wedge f(y)
\end{equation}

\section*{Partial Redundancy Elimination}

\section*{"Potentially Available" expressions}
An expression \(x + y\) is "potentially available" if two criterion are met
\begin{enumerate}
    \item[1] It is available in the usual sense, has already been computed
    \item[2] It is anticipated. It could be available if we choose to compute it here.
\end{enumerate}
\section*{Single Static Assignment}

In SSA, every variable is assigned a value at most once. For every reassignment
of some variable, we create a fresh version of the variable on in LHS. At a 
"join point" in the CFG, we use a special function called the Phi function of 
merge multiple definitions of a variable into a single definition.\\

\subsection*{Placement of \(\Phi\) functions}
Knowing where to place a \(\Phi\) function and doing it efficiently is not a
trivial task.

\subsubsection*{Naive SSA}
At each join point, insert a \(\Phi\) function for all live variables. However
some variables may not have been redefined at the join point, thus this algorithm
may produce unnecessary \(\Phi\) functions

\subsubsection*{Minimal SSA}
At each join point, insert a \(\Phi\) function for all live variables with
multiple outstanding definitions. The algorithm for this requires computing the
\textbf{Interated Dominance Frontier} of each definition.

\begin{enumerate}
    \item \textbf{Dominator} - A node \(n\) dominates another node \(m\) if
          every path from the entry to \(m\) must go through \(n\)
    \item \textbf{Strict Dominator} - A node \(n\) strictly dominates another
          node \(m\) if \(n\) dominates \(m\) and \(n \neq m\)
    \item \textbf{Dominance Frontier} - The dominance frontier of a node \(n\)
          is the set of all nodes \(m\) such that \(m\) has a predecessor that
          is dominated by \(n\) and \(n\) does not strictly dominate \(m\)
          itself (i.e \(n \neq m\)).
\end{enumerate}

\fbox{
    \parbox{\textwidth}{
        Place a \(\Phi\) function in the dominance frontier of every SSA variable
        definition
    }

}

\subsection*{Interated Dominance Frontier Algorithm}



\end{document}